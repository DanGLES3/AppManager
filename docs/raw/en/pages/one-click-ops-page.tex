% SPDX-License-Identifier: GPL-3.0-or-later OR CC-BY-SA-4.0
\section{1-Click Ops Page}\label{sec:1-click-ops-page} %%##$section-title>>
%%!!intro<<
This page is displayed on selecting the \hyperref[subsubsec:main:1-click-ops]{1-Click Ops option} in the \hyperref[subsec:main-page-options-menu]{main menu}.
%%!!>>

\subsection{Block/Unblock Trackers}\label{subsec:block-unblock-trackers} %%##$block-unblock-trackers-title>>
%%!!block-unblock-trackers<<
This option can be used to block or unblock the ad/tracker components from the installed applications.
On selecting this option, App Manager will ask if it should list trackers from all the applications or only from the user applications.
Novice users should avoid blocking trackers from the system applications in order to avoid bad consequences.
After that, a multi-choice dialog box will appear where it is possible to exclude one or more applications from this operation.
The changes are applied immediately on pressing the \textit{block} or \textit{unblock} button.

\begin{warning}{Notice}
    Certain applications may not function as expected after blocking their trackers.
    If that is the case, remove the blocking rules all at once or one by one in the component tabs of the \hyperref[sec:app-details-page]{App Details page} for the corresponding application.
\end{warning}

\seealsoinline{\hyperref[par:appdetails:blocking-trackers]{App Details Page: Blocking Trackers}}
%%!!>>

\subsection{Block Components\dots}\label{subsec:block-components-dots} %%##$block-components-dots-title>>
%%!!block-components-dots<<
This option can be used to block certain application components as specified by their signatures.
A signature of a component is the full name or partial name of the component.
For safety, it is recommended to add a \texttt{.} (dot) at the end of each partial signature, because the underlying
algorithm searches and matches the components in a greedy manner.
It is also possible to insert more than one signature in which case all the signatures have to be separated by white spaces.
Similar to the option above, there is also an option to apply blocking to the system applications.

\begin{danger}{Caution}
    If you are not aware of the consequences of blocking applcations components by their signatures, you should avoid
    using this option as it may result in bootloop or soft brick, and you may have to apply factory reset as a result.
\end{danger}
%%!!>>

\subsection{Set Mode for App Ops\dots}\label{subsec:set-mode-for-app-ops-dots} %%##$set-mode-for-app-ops-dots-title>>
%%!!set-mode-for-app-ops-dots<<
This option can be used to configure certain \hyperref[ch:app-ops]{applcation operations} of all or selected applications.
There are two fields. The first field can be used to insert more than one app op constants (either names or values) separated by white spaces.
It is not always possible to know in advance about all the app op constants as they vary from device to device and from OS to OS\@.
Desired app op constant can be found in the \textit{App Ops} tab located in the \hyperref[sec:app-details-page]{App Details page}.
The second field can be used to insert or select one of the \hyperref[subsec:mode-constants]{modes} that will be set against the specified app ops.

\begin{danger}{Caution}
    Unless you are well-informed about app ops and the consequences of blocking them, you should avoid using this option.
\end{danger}
%%!!>>

\subsection{Back up}\label{subsec:1-click-back-up} %%##$1-click-back-up-title>>
%%!!back-up<<
1-Click options for back up. As a precaution, it lists the affected backups before performing any operation.

\paragraph{Back up all apps.} Back up all the installed applications.

\paragraph{Redo existing backups.} Back up all the installed applications that have a previous backup.

\paragraph{Back up apps without backups.} Back up all the installed applications without a previous backup.

\paragraph{Verify and redo backups.} Verify the recently made backups of the installed applications and redo backup if necessary.

\paragraph{Back up apps with changes.} If an app has changed since the last backup, redo its backup.
It checks a number of indices including application version, last update date, last launch date, integrity and file hashes.
Directory hashes are taken during the backup process and are stored in a database.
On running this operation, new hashes are taken and compared with the ones kept in the database.
%%!!>>

\subsection{Restore}\label{subsec:1-click-restore} %%##$1-click-restore-title>>
%%!!restore<<
1-Click options for restore. As a precaution, it lists the affected backups before performing any operation.

\paragraph{Restore all apps.} Restore \textit{base backup} of all the backed up applications.

\paragraph{Restore not installed apps.} Restore \textit{base backup} of all the backed up applications that are not currently installed.

\paragraph{Restore latest backups.} Restore \textit{base backup} of already installed applications whose version codes are higher than the installed version code.
%%!!>>

\subsection{Trim Caches in All Apps}\label{subsec:trim-caches-in-all-apps} %%##$trim-caches-in-all-apps-title>>
%%!!trim-caches-in-all-apps<<
Delete caches from all applications, including Android system. During this operation, caches of all the running
applications may not be cleared as expected.
%%!!>>
